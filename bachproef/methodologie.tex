%%=============================================================================
%% Methodologie
%%=============================================================================

\chapter{Methodologie}
\label{ch:methodologie}

%% TODO: Hoe ben je te werk gegaan? Verdeel je onderzoek in grote fasen, en
%% licht in elke fase toe welke stappen je gevolgd hebt. Verantwoord waarom je
%% op deze manier te werk gegaan bent. Je moet kunnen aantonen dat je de best
%% mogelijke manier toegepast hebt om een antwoord te vinden op de
%% onderzoeksvraag.

Om de doelstelling van het onderzoek concreet te maken, werd volgende onderzoeksvraag geformuleerd:

\begin{center}
	\textit{\textbf{``Hoe kunnen softwareontwikkelaars overgaan tot integratie van blockchaintechnologie voor de dataopslag van hun ERP-systeem?''}}
\end{center}

Om op een systematische manier een antwoord te vinden op deze onderzoeksvraag, werd ze ontleed in vier deelvragen:

\begin{enumerate}
	\item Wat is een blockchain?
	\item Hoe wordt de data van het ERP-syteem opgeslagen? (\textit{as is})
	\item Welke mogelijkheden bieden blockchains voor een ERP-systeem?
	\item Hoe kunnen softwareontwikkelaars de data van hun ERP-systeem opslaan in een blockchain? (\textit{to be})
\end{enumerate}

Al van bij aanvang was het de bedoeling om, vertrekkende vanuit een high-level, zo snel mogelijk toe te werken naar de concrete bedrijfscasus. Die benadering weerspiegelt zich in de opgesomde deelvragen en de volgorde ervan.
Aan elk van de deelvragen werd een eigen hoofdstuk toegewijd.

Als eerste fase van het onderzoek, was het van cruciaal belang om een degelijk inzicht te verwerven in het onderwerp ``blockchain''. Dat gebeurde aan de hand van een \textbf{literatuurstudie} rond de eerste deelvraag van het onderzoek.
\textbf{Hoofdstuk~\ref{ch:blockchain} -- \nameref{ch:blockchain}} brengt de resultaten van deze studie samen, om de lezer zo de nodige achtergrond voor dit thema te bieden. Een rudimentaire implementatie van een blockchain wordt stap per stap opgebouwd, als oplossing voor een voorgedefiniëerde probleemstelling. Met deze opbouw komt de werking duidelijker naar voren. Dat is nodig om een gegrond inzicht te krijgen in de essentiële kenmerken van blockchains.

In \textbf{Hoofdstuk~\ref{ch:cpsolution} -- \nameref{ch:cpsolution}} werd het ERP-systeem van 14IT onder de loep genomen. Op die manier behandelt het de tweede deelvraag. Aan de hand van een \textbf{\textit{case study}} werd de huidige dataopslag van CPSolution in kaart gebracht. Daarmee werd het ``speelveld'' voor deze bachelorproef geconcretiseerd. Dat maakte het mogelijk om beter in te schatten welke noden, kansen en aandachtspunten er bestaan op vlak van dataopslag. Het was belangrijk om deze as-issituatie voldoende te kennen, om zo gericht naar antwoorden op de volgende deelvraag te zoeken.

De voorgaande twee hoofdstukken leggen de basis voor de derde deelvraag. Deze wordt behandeld in \textbf{Hoofdstuk~\ref{ch:blockchaintoepassingen-voor-erp} -- \nameref{ch:blockchaintoepassingen-voor-erp}}. Aangezien zowel het begrip ``blockchain'' als ``ERP'' nu gekend is, kan gezocht worden naar een synergie tussen beiden. Twee toepassingen uit de Gartner Hype Cycle\footnote{Zie Sectie~\ref{sec:verantwoording} -- \nameref{sec:verantwoording}.} geven de aanleiding tot NFT's, zo werd duidelijk uit een korte \textbf{literatuurverkenning}. Het kernidee dat tot de uiteindelijke proof of concept leidde, is dat het mogelijk is om allerlei data in een blockchain op te nemen. Dat wordt geïllustreerd met een concreet voorbeeld van een NFT.

Een \textbf{proof of concept} biedt het antwoord op de laatste deelvraag van het onderzoek. \textbf{Hoofdstuk~\ref{ch:proof-of-concept} -- \nameref{ch:proof-of-concept}} werd hieraan toegewijd. In dit hoofdstuk wordt een praktisch voorbeeld uitgewerkt van een blockchaintoepassing voor CPSolution. Hiervoor werd gesteund op de services van mintBlue, een bedrijf dat naar boven kwam bij een \textbf{marktverkenning} van het gekozen onderwerp.

Tenslotte, wordt in \textbf{Hoofdstuk~\ref{ch:conclusie} -- \nameref{ch:conclusie}} een antwoord geformuleerd op de overkoepelende onderzoeksvraag. 

