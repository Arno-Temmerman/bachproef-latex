%%=============================================================================
%% Samenvatting
%%=============================================================================

% TODO: De "abstract" of samenvatting is een kernachtige (~ 1 blz. voor een
% thesis) synthese van het document.
%
% Deze aspecten moeten zeker aan bod komen:
% - Context: waarom is dit werk belangrijk?
% - Nood: waarom moest dit onderzocht worden?
% - Taak: wat heb je precies gedaan?
% - Object: wat staat in dit document geschreven?
% - Resultaat: wat was het resultaat?
% - Conclusie: wat is/zijn de belangrijkste conclusie(s)?
% - Perspectief: blijven er nog vragen open die in de toekomst nog kunnen
%    onderzocht worden? Wat is een mogelijk vervolg voor jouw onderzoek?
%
% Max 1 bladzijde
%
% LET OP! Een samenvatting is GEEN voorwoord!

%%---------- Samenvatting -----------------------------------------------------

\chapter*{Samenvatting}

Met het ontstaan van de \textit{Bitcoin} werd de IT-sector plots een hypetechnologie rijker: de blockchain. De techniek die oorspronkelijk enkel gebruikt werd voor digitale munten, maakt ondertussen ook in allerlei andere branches haar intrede. In het licht van \textit{continuous improvement} gaan ontwikkelaars op zoek naar manieren om dit nieuwe gegeven een plaats te geven binnen hun systemen. Op die manier wensen ze relevant te blijven en eventueel een voorsprong te krijgen op concurrenten. Dat is ook het geval bij 14IT. Het softwarebedrijf ontwikkelde in 2014 een ERP-systeem voor kmo's, genaamd CPSolution. Eigenaar en co-promotor Geert Borloo stelt zich de vraag of er in dit systeem een \textit{use case} weggelegd is voor blockchains. Van hieruit ontstond de nood aan een onderzoek naar een mogelijke synergie tussen blockchain en ERP.

De interesse in blockchains is niet onterecht. De technologie brengt, met haar typerende eigenschappen, een aantal nieuw mogelijkheden ter tafel. Gezien het hier een hypetechnologie betreft, is het echter heel gemakkelijk om het potentieel ervan te overschatten. Daardoor is het niet evident om door het bos de bomen te zien. Met dit onderzoek werd, doorheen alle hype, gericht gezocht naar een rendabele blockchaintoepassing voor CPSolution. Na een literatuurstudie, case study en marktverkenning werd die toepassing ook gevonden. Het eindresultaat, een proof of concept, toont niet alleen aan dat het daadwerkelijk mogelijk is om als kmo ERP-data (facturen) op te slaan in een blockchain. Het illustreert ook dat hier een echte \textit{use case} voor weggelegd is, genaamd NFT invoicing.

Dit eindwerk dient --in het bijzonder voor 14IT-- als naslagwerk om de conclusies van het onderzoek vast te stellen. De bedoeling is (en was) dat het bedrijf er op die manier de vruchten van kan plukken. Dat kan door het principe uit de proof of concept te implementeren als nieuwe module van CPSolution.

