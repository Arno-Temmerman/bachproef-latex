%---------- Inleiding ---------------------------------------------------------

\section{Introductie} % The \section*{} command stops section numbering
\label{sec:introductie}

In 1982 introduceerde cryptograaf David Chaum een protocol dat de grondslag vormde van wat men vandaag een blockchain noemt. De eerste werkzame implementatie werd pas een feit toen Satoshi Nakamoto in 2008 de eerste gedecentraliseerde digitale munt \textit{Bitcoin} in het leven riep. Sindsdien wordt de term ``blockchain'' alom gebruikt als buzzwoord voor alles wat met \textit{cryptocurrencies} te maken heeft. Zo ziet men echter enkel het topje van de ijsberg. Ook in andere branches ziet men namelijk steeds meer opportuniteiten voor deze nieuwe vorm van dataopslag. De verwachting is dat blockchain nog meer te bieden heeft dan wat op dit moment al gekend is. Daarom zoekt men vanuit allerlei frisse invalshoeken naar vernieuwende toepassingen voor deze technologie. Ondernemers stellen zich de vraag of blockchain ook voor hun een meerwaarde in petto heeft. Dit is ook het geval bij softwareontwikkelaar 14IT.

14IT ontwikkelde het ERP-syteem ``CPSolutions'' voor zijn klanten. Een deel van het takenpakket bestaat uit het verder ontwikkelen en onderhouden van deze software. In dit kader van continuous improvement ontstond de interesse in dit onderwerp. De vraag stelt zich of het programma verbeterd of uitgebreid kan worden door in te zetten op blockchain.

Aangezien het onderwerp van hieruit werd aangereikt, zal deze bedrijfscasus ook de rode draad vormen die door het werkstuk loopt. Hoewel het thema \textit{blockchain for ERP} in eerste instantie vanop \textit{high-level} benaderd zal worden, is het de bedoeling om doorheen het proces steeds concreter te gaan. Zo hoort uiteindelijk een resultaat behaald te worden dat praktisch inzetbaar is, in het bijzonder door 14IT.


De doelstelling, zoals hierboven beschreven, kan herleid worden naar volgende onderzoeksvraag:

\begin{center}
	\textit{\textbf{``Hoe kunnen softwareontwikkelaars overgaan tot integratie van blockchaintechnologie voor de dataopslag van een ERP-systeem?''}}
\end{center}

In sectie \ref{sec:methodologie} - \nameref{sec:methodologie} staat beschreven hoe ik deze onderzoeksvraag op een systematische manier zal trachten te beantwoorden.




%---------- Stand van zaken ---------------------------------------------------

\section{State-of-the-art}
\label{sec:state-of-the-art}

Hier beschrijf je de \emph{state-of-the-art} rondom je gekozen onderzoeksdomein. Dit kan bijvoorbeeld een literatuurstudie zijn. Je mag de titel van deze sectie ook aanpassen (literatuurstudie, stand van zaken, enz.). Zijn er al gelijkaardige onderzoeken gevoerd? Wat concluderen ze? Wat is het verschil met jouw onderzoek? Wat is de relevantie met jouw onderzoek?

Verwijs bij elke introductie van een term of bewering over het domein naar de vakliteratuur, bijvoorbeeld~\autocite{Doll1954}! Denk zeker goed na welke werken je refereert en waarom.

% Voor literatuurverwijzingen zijn er twee belangrijke commando's:
% \autocite{KEY} => (Auteur, jaartal) Gebruik dit als de naam van de auteur
%   geen onderdeel is van de zin.
% \textcite{KEY} => Auteur (jaartal)  Gebruik dit als de auteursnaam wel een
%   functie heeft in de zin (bv. ``Uit onderzoek door Doll & Hill (1954) bleek
%   ...'')

Je mag gerust gebruik maken van subsecties in dit onderdeel.

%---------- Methodologie ------------------------------------------------------
\section{Methodologie}
\label{sec:methodologie}

De onderzoeksvraag (zoals beschreven in sectie \ref{sec:introductie} - \nameref{sec:introductie}) geeft de essentie van het onderzoek weer. Het systematisch beantwoorden van onderstaande \textbf{deelvragen} zou mij in staat moeten stellen om de bijhorende doelstelling te behalen.

\begin{enumerate}
	\item Wat is een blockchain?
	\item Hoe wordt de data van een ERP-syteem opgeslagen? (AS IS)
	\item Welke mogelijkheden biedt blockchain voor een ERP-systeem?
	\item Hoe kunnen softwareontwikkelaars de data van hun ERP-systeem opslaan in een blockchain? (TO BE)
	
\end{enumerate}

Zoals hiervoor beschreven is het de bedoeling om, vertrekkende vanuit een \textit{high level approach}, steeds concreter toe te werken naar de concrete bedrijfscasus. Die benadering weerspiegelt zich in de opgesomde deelvragen en de volgore waarin die elkaar opvolgen. Een \textbf{literatuurstudie} vormt de basis van elk van deze deelvragen en is daarom de eerste fase van het onderzoek. Hierin kan uitgespit worden
\begin{enumerate}
	\item hoe de werking en implementatie van een blockchain eruit ziet; welke soorten blockchains bestaan;
	\item welke eisen en aandachtspunten er zijn qua dataopslag in een ERP-context; welke noden en kansen er nog bestaan bij het opslaan van data omtrent de supply chain; use cases van blockchains in andere bedrijven;
	\item welke voor- en nadelen een blockchain met zich meebrengt; 
	\item welke manieren er bestaan om data vast te leggen in een blockchain; welke kosten er gepaard gaan bij deze integratie.
\end{enumerate}

Het spreekt voor zich dat lauter een literatuurstudie niet volstaat om alle deelvragen naar behoren te behandelen. 
De tweede deelvraag biedt zich aan als gelegenheid om de \textit{as-is} situatie in kaart te brengen. Een \textbf{\textit{case study}} van \textit{CPSolutions} is hier wel op zijn plaats. Ook dient onderzocht te worden welke mogelijkheden al gebruikt worden op de markt en voor welke use cases. Dit kan aan de hand van een soort kleinschalige \textbf{marktanalyse}, horend bij de derde deelvraag. Op dat moment zou het mogelijk moeten zijn om de inschatting te maken of blockchain een meerwaarde kan bieden. In de laatste deelvraag proberen we de \textit{to-be} situatie in kaart te brengen.


%---------- Verwachte resultaten ----------------------------------------------
\section{Verwachte resultaten}
\label{sec:verwachte_resultaten}

Zoals al eerder vermeld, moet het onderzoek resulteren in iets wat praktisch bruikbaar is binnen de softwareontwikkeling. 

Enerzijds zal het resulterend \textbf{werkstuk} de lezer in staat stellen om zich vlot in te werken in het onderwerp. Aan de hand van de werking, \textit{pros and cons}, opportuniteiten en platformen van \textit{blockchain for ERP} die hierin beschreven zullen worden, kan men beslissen of men wilt overgaan tot integratie. Er valt voorlopig al veel te vinden over blockchain, waarvan niet alles even interessant is in de context van een ERP-systeem. Het eindresultaat moet softwareontwikkelaars helpen om door de bomen het bos te zien.

Anderzijds moet er ook iets voorzien worden voor die softwareontwikkelaars die de stap effectief wensen te nemen. Om een houvast te bieden tijdens die overgang zou een vooropgestelde \textbf{procedure} uitgewerkt kunnen worden. Beeld hierbij een soort plan (van aanpak) in dat de ondernemer helpt om stap per stap naar de \textit{to-be} situatie toe te werken. Het kostenplaatje van heel dit gegeven mag hierin niet ontbreken.


%---------- Verwachte conclusies ----------------------------------------------
\section{Verwachte conclusies}
\label{sec:verwachte_conclusies}

Hier beschrijf je wat je verwacht uit je onderzoek, met de motivatie waarom. Het is \textbf{niet} erg indien uit je onderzoek andere resultaten en conclusies vloeien dan dat je hier beschrijft: het is dan juist interessant om te onderzoeken waarom jouw hypothesen niet overeenkomen met de resultaten.

