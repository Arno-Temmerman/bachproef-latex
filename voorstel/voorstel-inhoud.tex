%---------- Inleiding ---------------------------------------------------------

\section{Introductie} % The \section*{} command stops section numbering
\label{sec:introductie}

In 1982 introduceerde cryptograaf David Chaum een protocol dat de grondslag vormde van wat men vandaag een blockchain noemt. De eerste werkzame implementatie werd pas een feit toen Satoshi Nakamoto in 2008 de eerste gedecentraliseerde digitale munt \textit{Bitcoin} in het leven riep. Sindsdien wordt de term ``blockchain'' alom gebruikt als buzzwoord voor alles wat met \textit{cryptocurrencies} te maken heeft. Zo ziet men echter enkel het topje van de ijsberg. Ook in andere branches ziet men namelijk steeds meer opportuniteiten voor deze nieuwe vorm van dataopslag. De verwachting is dat blockchain nog meer te bieden heeft dan wat op dit moment al gekend is. Daarom zoekt men vanuit allerlei frisse invalshoeken naar vernieuwende toepassingen voor deze technologie. Ondernemers stellen zich de vraag of blockchain ook voor hun een meerwaarde in petto heeft. Dit is ook het geval bij softwareontwikkelaar 14IT.

14IT ontwikkelde het ERP-syteem ``CPSolutions'' voor zijn klanten. Een deel van het takenpakket bestaat uit het verder ontwikkelen en onderhouden van deze software. In dit kader van continuous improvement ontstond de interesse in dit onderwerp. De vraag stelt zich of het programma verbeterd of uitgebreid kan worden door in te zetten op blockchain.

Aangezien het onderwerp van hieruit werd aangereikt, zal deze bedrijfscasus ook de rode draad vormen die door het werkstuk loopt. Hoewel het thema \textit{blockchain for ERP} in eerste instantie vanop \textit{high-level} benaderd zal worden, is het de bedoeling om doorheen het proces steeds concreter te gaan. Zo hoort uiteindelijk een resultaat behaald te worden dat praktisch inzetbaar is, in het bijzonder door 14IT.


De doelstelling, zoals hierboven beschreven, kan herleid worden naar volgende onderzoeksvraag:

\begin{center}
	\textit{\textbf{``Hoe kunnen softwareontwikkelaars overgaan tot integratie van blockchaintechnologie voor de dataopslag van een ERP-systeem?''}}
\end{center}

In sectie \ref{sec:methodologie} - \nameref{sec:methodologie} staat beschreven hoe ik deze onderzoeksvraag op een systematische manier zal trachten te beantwoorden.




%---------- Stand van zaken ---------------------------------------------------

\section{State-of-the-art}
\label{sec:state-of-the-art}

Hier beschrijf je de \emph{state-of-the-art} rondom je gekozen onderzoeksdomein. Dit kan bijvoorbeeld een literatuurstudie zijn. Je mag de titel van deze sectie ook aanpassen (literatuurstudie, stand van zaken, enz.). Zijn er al gelijkaardige onderzoeken gevoerd? Wat concluderen ze? Wat is het verschil met jouw onderzoek? Wat is de relevantie met jouw onderzoek?

Verwijs bij elke introductie van een term of bewering over het domein naar de vakliteratuur, bijvoorbeeld~\autocite{Doll1954}! Denk zeker goed na welke werken je refereert en waarom.

% Voor literatuurverwijzingen zijn er twee belangrijke commando's:
% \autocite{KEY} => (Auteur, jaartal) Gebruik dit als de naam van de auteur
%   geen onderdeel is van de zin.
% \textcite{KEY} => Auteur (jaartal)  Gebruik dit als de auteursnaam wel een
%   functie heeft in de zin (bv. ``Uit onderzoek door Doll & Hill (1954) bleek
%   ...'')

Je mag gerust gebruik maken van subsecties in dit onderdeel.

%---------- Methodologie ------------------------------------------------------
\section{Methodologie}
\label{sec:methodologie}

Hier beschrijf je hoe je van plan bent het onderzoek te voeren. Welke onderzoekstechniek ga je toepassen om elk van je onderzoeksvragen te beantwoorden? Gebruik je hiervoor experimenten, vragenlijsten, simulaties? Je beschrijft ook al welke tools je denkt hiervoor te gebruiken of te ontwikkelen.

%---------- Verwachte resultaten ----------------------------------------------
\section{Verwachte resultaten}
\label{sec:verwachte_resultaten}

Hier beschrijf je welke resultaten je verwacht. Als je metingen en simulaties uitvoert, kan je hier al mock-ups maken van de grafieken samen met de verwachte conclusies. Benoem zeker al je assen en de stukken van de grafiek die je gaat gebruiken. Dit zorgt ervoor dat je concreet weet hoe je je data gaat moeten structureren.

%---------- Verwachte conclusies ----------------------------------------------
\section{Verwachte conclusies}
\label{sec:verwachte_conclusies}

Hier beschrijf je wat je verwacht uit je onderzoek, met de motivatie waarom. Het is \textbf{niet} erg indien uit je onderzoek andere resultaten en conclusies vloeien dan dat je hier beschrijft: het is dan juist interessant om te onderzoeken waarom jouw hypothesen niet overeenkomen met de resultaten.

