%==============================================================================
% Sjabloon onderzoeksvoorstel bachelorproef
%==============================================================================
% Gebaseerd op LaTeX-sjabloon ‘Stylish Article’ (zie voorstel.cls)
% Auteur: Jens Buysse, Bert Van Vreckem
%
% Compileren in TeXstudio:
%
% - Zorg dat Biber de bibliografie compileert (en niet Biblatex)
%   Options > Configure > Build > Default Bibliography Tool: "txs:///biber"
% - F5 om te compileren en het resultaat te bekijken.
% - Als de bibliografie niet zichtbaar is, probeer dan F5 - F8 - F5
%   Met F8 compileer je de bibliografie apart.
%
% Als je JabRef gebruikt voor het bijhouden van de bibliografie, zorg dan
% dat je in ``biblatex''-modus opslaat: File > Switch to BibLaTeX mode.

\documentclass{voorstel}

\usepackage{lipsum}

%------------------------------------------------------------------------------
% Metadata over het voorstel
%------------------------------------------------------------------------------

%---------- Titel & auteur ----------------------------------------------------

% TODO: geef werktitel van je eigen voorstel op
\PaperTitle{Titel voorstel}
\PaperType{Onderzoeksvoorstel Bachelorproef 2019-2020} % Type document

% TODO: vul je eigen naam in als auteur, geef ook je emailadres mee!
\Authors{Steven Stevens\textsuperscript{1}} % Authors
\CoPromotor{Piet Pieters\textsuperscript{2} (Bedrijfsnaam)}
\affiliation{\textbf{Contact:}
  \textsuperscript{1} \href{mailto:steven.stevens.u1234@student.hogent.be}{steven.stevens.u1234@student.hogent.be};
  \textsuperscript{2} \href{mailto:piet.pieters@acme.be}{piet.pieters@acme.be};
}

%---------- Abstract ----------------------------------------------------------

\Abstract{Hier schrijf je de samenvatting van je voorstel, als een doorlopende tekst van één paragraaf. Wat hier zeker in moet vermeld worden: \textbf{Context} (Waarom is dit werk belangrijk?); \textbf{Nood} (Waarom moet dit onderzocht worden?); \textbf{Taak} (Wat ga je (ongeveer) doen?); \textbf{Object} (Wat staat in dit document geschreven?); \textbf{Resultaat} (Wat verwacht je van je onderzoek?); \textbf{Conclusie} (Wat verwacht je van van de conclusies?); \textbf{Perspectief} (Wat zegt de toekomst voor dit werk?).

Bij de sleutelwoorden geef je het onderzoeksdomein, samen met andere sleutelwoorden die je werk beschrijven.

Vergeet ook niet je co-promotor op te geven.
}

%---------- Onderzoeksdomein en sleutelwoorden --------------------------------
% TODO: Sleutelwoorden:
%
% Het eerste sleutelwoord beschrijft het onderzoeksdomein. Je kan kiezen uit
% deze lijst:
%
% - Mobiele applicatieontwikkeling
% - Webapplicatieontwikkeling
% - Applicatieontwikkeling (andere)
% - Systeembeheer
% - Netwerkbeheer
% - Mainframe
% - E-business
% - Databanken en big data
% - Machineleertechnieken en kunstmatige intelligentie
% - Andere (specifieer)
%
% De andere sleutelwoorden zijn vrij te kiezen

\Keywords{Onderzoeksdomein. Keyword1 --- Keyword2 --- Keyword3} % Keywords
\newcommand{\keywordname}{Sleutelwoorden} % Defines the keywords heading name

%---------- Titel, inhoud -----------------------------------------------------

\begin{document}

\flushbottom % Makes all text pages the same height
\maketitle % Print the title and abstract box
\tableofcontents % Print the contents section
\thispagestyle{empty} % Removes page numbering from the first page

%------------------------------------------------------------------------------
% Hoofdtekst
%------------------------------------------------------------------------------

% De hoofdtekst van het voorstel zit in een apart bestand, zodat het makkelijk
% kan opgenomen worden in de bijlagen van de bachelorproef zelf.
%---------- Inleiding ---------------------------------------------------------

\section{Introductie} % The \section*{} command stops section numbering
\label{sec:introductie}

In 1982 introduceerde cryptograaf David Chaum een protocol dat de grondslag vormde van wat men vandaag een blockchain noemt. De eerste werkzame implementatie werd pas een feit toen Satoshi Nakamoto in 2008 de eerste gedecentraliseerde digitale munt \textit{Bitcoin} in het leven riep. Sindsdien wordt de term ``blockchain'' alom gebruikt als buzzwoord voor alles wat met \textit{cryptocurrencies} te maken heeft. Zo ziet men echter enkel het topje van de ijsberg. Ook in andere branches ziet men namelijk steeds meer opportuniteiten voor deze nieuwe vorm van dataopslag. De verwachting is dat blockchain nog meer te bieden heeft dan wat op dit moment al gekend is. Daarom zoekt men vanuit allerlei frisse invalshoeken naar vernieuwende toepassingen voor deze technologie. Ondernemers stellen zich de vraag of blockchain ook voor hun een meerwaarde in petto heeft. Dit is ook het geval bij softwareontwikkelaar 14IT. In het verdere voorstel en onderzoek zal aan bod komen wat deze ``meerwaarde'' juist inhoudt.

14IT ontwikkelde het ERP-syteem ``CPSolution'' voor zijn klanten (kmo's). Een deel van het takenpakket bestaat uit het verder ontwikkelen en onderhouden van deze software. In dit kader van continuous improvement ontstond de interesse in dit onderwerp. De vraag stelt zich of het programma verbeterd of uitgebreid kan worden door in te zetten op blockchain.

Aangezien het onderwerp van hieruit werd aangereikt, zal deze bedrijfscasus ook de rode draad vormen die door het werkstuk loopt. Hoewel het thema \textit{blockchain for ERP} in eerste instantie vanop \textit{high-level} benaderd zal worden, is het de bedoeling om doorheen het proces steeds concreter te gaan. Zo hoort uiteindelijk een resultaat behaald te worden dat praktisch inzetbaar is, in het bijzonder door 14IT.


De doelstelling, zoals hierboven beschreven, kan herleid worden naar volgende onderzoeksvraag:

\begin{center}
	\textit{\textbf{``Hoe kunnen softwareontwikkelaars overgaan tot integratie van blockchaintechnologie voor de dataopslag van een ERP-systeem?''}}
\end{center}

In sectie \ref{sec:methodologie} - \nameref{sec:methodologie} staat beschreven hoe ik deze onderzoeksvraag op een systematische manier zal trachten te beantwoorden.




%---------- Stand van zaken ---------------------------------------------------
\section{Literatuurstudie}
\label{sec:state-of-the-art}

\subsection{Wat is blockchain?}
\label{sub:wat-is-blockchain}

Een blockchain kan voorgesteld worden als een keten van datablokken, waarin nieuwe data enkel kan toegevoegd worden onder de vorm van een nieuw blok aan het einde van de keten. Het is een techniek om transacties zodanig bij te houden dat deze later niet meer vervalst kunnen worden. Deze eigenschap, genaamd \textit{immutability}, wordt mogelijk gemaakt door het gedecentraliseerde karakter van de blokketen. Het is namelijk zo dat de data wordt opgeslagen in een speciale, gedistribueerde database die men een \textit{distributed ledger} noemt. Hierin worden de transacties verspreid over verschillende toestellen (\textit{nodes}) waarvoor geen centrale autoriteit bestaat. Participanten hoeven de betrouwbaarheid van tussenpersonen of andere deelnemers niet meer te beoordelen, aangezien de betrouwbaarheid van het systeem als geheel al volstaat~\autocite{Nofer2017}. Die betrouwbaarheid wordt verwezenlijkt door de slimme implementatie van blockchain die gebruik maakt van cryptografie, iets wat aan bod zal komen in een uitgebreidere literatuurstudie (zie sectie \ref{sec:methodologie} - \nameref{sec:methodologie}).



\subsection{Perspectieven}
\label{sub:perspectieven}

In de oorspronkelijke toepassing stond een transactie voor de bitcoin-overdracht van een partij naar een andere~\autocite{Pierro2017}. 

Er zijn echter veel meer transacties denkbaar dan eenvoudige gelduitwisselingen. Andere voorbeelden zijn het ondertekenen van slimme contracten, uitwisselen van aandelen, obligaties of hypotheken enzovoort (Blockchain 2.0). Blockchain 3.0 zoekt zelfs naar toepassingen binnen de overheid, wetenschap en kunst~\autocite{Swan2015}.

University of Cambridge gebruikte data van meer dan 200 start-ups, financiële ondernemingen, banken en andere publieke instituties om na te gaan welke toepassingen al bestaan voor \textit{distributed ledger} technologie. Men kwam tot een lijst van 132 use cases, gegroepeerd per sector. Hoewel men een grote aanwezigheid vaststelde van banken en betalingen, stelde men een toenemende interesse vast voor toepassingen zoals de \textit{supply chain}~\autocite{Hileman2017}.

Rejeb ziet een kans om ERP-systemen van verschillende ontwikkelaars en stakeholders samen te brengen op eenzelfde platform. Blockchain zou een dergelijk ERP-netwerk mogelijk kunnen maken~\autocite{Rejeb2018}.

Zoals eerder vermeld is een blockchain \textit{immutable} en gedecentraliseerd. Use cases voor het ERP-systeem van een kmo zullen een meerwaarde bieden, wanneer deze twee eigenschappen benut worden. Zo kan men bijvoorbeeld transacties omtrent facturen opnemen in een \textit{distributed ledger}. Een klant die dan een digitale factuur ontvangt, kan dankzij de \textit{immutability} nagaan of deze effectief van de afzender komt en dus geen \textit{spoofing} is.


\subsection{State-of-the-art}
\label{sub:state-of-the-art}

De huidige platformen voor blockchain staan nog in hun kinderschoenen. Nog zo recent als 2016 waren blockchain technologieën, die voor de markt bestemd waren, in een experimentele fase~\autocite{Davidson2016}.

Ondertussen zijn er al spelers die blockchainplatformen aanbieden op de markt (zoals mintBlue). Deze ondernemingen worstelden lang met uitdagingen rond privacy en schaalbaarheid, waarvoor mogelijke oplossingen bedacht werden~\autocite{Kaptijn}. Er zijn al enkele voorbeelden van ondernemingen die mee op de kar springen, zoals aanbieder van online boekhoudsoftware Yuki. Ook binnen de Vlaamse overheid zijn projecten lopend voor verschillende blockchain-toepassingen~\autocite{Schiltz2018}.


% Voor literatuurverwijzingen zijn er twee belangrijke commando's:
% \autocite{KEY} => (Auteur, jaartal) Gebruik dit als de naam van de auteur
%   geen onderdeel is van de zin.
% \textcite{KEY} => Auteur (jaartal)  Gebruik dit als de auteursnaam wel een
%   functie heeft in de zin (bv. ``Uit onderzoek door Doll & Hill (1954) bleek
%   ...'')


%---------- Methodologie ------------------------------------------------------
\section{Methodologie}
\label{sec:methodologie}

De onderzoeksvraag (zoals beschreven in sectie \ref{sec:introductie} - \nameref{sec:introductie}) geeft de essentie van het onderzoek weer. Het systematisch beantwoorden van onderstaande \textbf{deelvragen} zou mij in staat moeten stellen om de bijhorende doelstelling te behalen.

\begin{enumerate}
	\item Wat is een blockchain?
	\item Hoe wordt de data van een ERP-syteem opgeslagen? (\textit{as is})
	\item Welke mogelijkheden biedt blockchain voor een ERP-systeem?
	\item Hoe kunnen softwareontwikkelaars de data van het ERP-systeem opslaan in een blockchain? (\textit{to be})
	
\end{enumerate}

Zoals hiervoor beschreven is het de bedoeling om, vertrekkende vanuit een \textit{high level}, zo snel mogelijk concreet toe te werken naar de bedrijfscasus. Die benadering weerspiegelt zich in de opgesomde deelvragen en de volgorde ervan.

Een \textbf{literatuurstudie} vormt de basis van elk van deze deelvragen en is daarom de eerste fase van het onderzoek. Hierin kan uitgespit worden
\begin{enumerate}
	\item hoe de werking en implementatie van een blockchain eruit ziet; welke soorten blockchains bestaan;
	\item welke eisen en aandachtspunten er zijn qua dataopslag in een ERP-context; welke noden en kansen er nog bestaan bij het opslaan van data omtrent de \textit{supply chain}; use cases van blockchains in andere bedrijven;
	\item welke voor- en nadelen een blockchain met zich meebrengt; 
	\item welke manieren er bestaan om data vast te leggen in een blockchain; welke kosten er gepaard gaan bij deze integratie.
\end{enumerate}

Het spreekt voor zich dat louter een literatuurstudie niet volstaat om alle deelvragen naar behoren te behandelen. 
De tweede deelvraag biedt zich aan als gelegenheid om de \textit{as is} situatie in kaart te brengen. Een \textbf{\textit{case study}} van \textit{CPSolution} is hier wel op zijn plaats. Ook dient onderzocht te worden welke mogelijkheden al gebruikt worden op de markt en voor welke use cases. Dit kan aan de hand van een soort kleinschalige \textbf{marktanalyse}, horend bij de derde deelvraag. Op dat moment zou het mogelijk moeten zijn om de inschatting te maken of blockchain een meerwaarde kan bieden. In de laatste deelvraag proberen we de \textit{to-be} situatie in kaart te brengen.


%---------- Verwachte resultaten ----------------------------------------------
\section{Verwachte resultaten}
\label{sec:verwachte_resultaten}

Zoals al eerder vermeld, moet het onderzoek resulteren in iets wat praktisch bruikbaar is binnen de softwareontwikkeling. Hier ligt dan ook de focus van de bachelorproef.

Enerzijds zal het resulterend \textbf{werkstuk} de lezer in staat stellen om zich vlot in te werken in het onderwerp. Aan de hand van de werking, \textit{pros and cons}, opportuniteiten en platformen van \textit{blockchain for ERP} die hierin beschreven zullen worden, kan men beslissen of men wil overgaan tot integratie. Er valt voorlopig al veel te vinden over blockchain, waarvan niet alles even interessant is in de context van een ERP-systeem. Het eindresultaat moet 14IT helpen om door de bomen het bos te zien.

Anderzijds moet er ook iets voorzien worden voor die softwareontwikkelaars die de stap effectief wensen te zetten. Om een houvast te bieden tijdens die overgang zou een vooropgestelde \textbf{procedure} uitgewerkt kunnen worden. Beeld hierbij een soort plan (van aanpak) in dat de ondernemer helpt om stap per stap naar de \textit{to-be} situatie toe te werken. Het kostenplaatje van heel dit gegeven mag hierin niet ontbreken.


%---------- Verwachte conclusies ----------------------------------------------
\section{Verwachte conclusies}
\label{sec:verwachte_conclusies}

De \emph{state-of-the-art} rondom het gekozen onderzoeksdomein laat vermoeden dat blockchain veel in petto heeft voor allerhande toepassingen. Ook voor ERP-systemen klinkt het veelbelovend.	Vermoedelijk zullen toepassingen op het niveau van Blockchain 2.0 momenteel het meest aan de orde zijn voor kmo's zoals 14IT. Dit wordt dan het scope-gebied van het onderzoek.

Het is hoogstwaarschijnlijk uitvoerbaar om de data rond ERP vast te leggen in een blockchain. Het is niet ondenkbaar dat dit ook een meerwaarde zal hebben. De mate waarin dit ook opweegt tegen de bijhorende investeringen in tijd en kapitaal vormt voorlopig een groter vraagteken. Daarom is het moeilijk om nu al in te schatten of die omschakeling naar blockchain ook rendabel is (voor 14IT). De uitwerking van deze bachelorproef zal daar hopelijk bij helpen.



%------------------------------------------------------------------------------
% Referentielijst
%------------------------------------------------------------------------------
% TODO: de gerefereerde werken moeten in BibTeX-bestand ``voorstel.bib''
% voorkomen. Gebruik JabRef om je bibliografie bij te houden en vergeet niet
% om compatibiliteit met Biber/BibLaTeX aan te zetten (File > Switch to
% BibLaTeX mode)

\phantomsection
\printbibliography[heading=bibintoc]

\end{document}
